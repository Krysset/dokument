\documentclass[11pt, noincludeaddress, nopagination]{classes/cthit}
\usepackage{titlesec}
\usepackage{hyperref}
\usepackage{verbatimbox}
\usepackage{tabularx}

\hypersetup{
    colorlinks=true,
    linkcolor=blue,
    filecolor=magenta,      
    urlcolor=cyan,
    pdftitle={Overleaf Example},
    pdfpagemode=FullScreen,
    }

%\titleformat{\paragraph}[hang]{\normalfont\normalsize\bfseries}{\theparagraph}{1em}{}
%\titlespacing*{\paragraph}{0pt}{3.25ex plus 1ex minus 0.2ex}{0.9em}

\graphicspath{ {images/} }

\begin{document}

\title{Nyckelavtal}
\authors{Uppdaterad 2022}

\makeheadfoot
\makesimpletitle
\vspace{-2 cm}
Observera att för att få tillbaka depositionen ska alla nycklar återlämnas. Vid försenad återlämning dras en avgift på 100 kronor, vilken ökar med 150 kronor per påbörjad ny läsperiod. Vid borttappad nyckel tillkommer en avgift på 1000 kronor.
Om nyckeln lämnas tillbaka skadad tillkommer en kostnad på 200 kronor.
\subsection*{Fylls i av nyckellånare:}

Förnamn + 'Nick': \hspace{4 cm}  Efternamn:
\vspace{0.8 cm}
~\\
\noindent\rule{\textwidth}{0.4pt}

Personnummer: \hspace{4.5 cm} Mobilnummer:
\vspace{0.8 cm}
~\\
\noindent\rule{\textwidth}{0.4pt}

Epost:  \hspace{6.1 cm} Signatur:
\vspace{0.8 cm}
~\\
\noindent\rule{\textwidth}{0.4pt}

Med denna signatur godkänner jag det nyckelavtal, samt bilaga till nyckelavtal, som finns på Teknologsektionen Informationsteknik vid Chalmers.\\\\\\
Kommitté:
\vspace{0.8 cm}

\subsection*{Fylls i av nyckelansvarig:}
Nyckelnummer: \hspace{4.7 cm}Utlämningsdatum:
\vspace{0.8 cm}~\\
\noindent\rule{\textwidth}{0.4pt}

Senast återlämning: \hspace{3.8 cm} Deposition betald:
\vspace{0.8 cm}~\\
\noindent\rule{\textwidth}{0.4pt}
Namn nyckelansvarig:  \hspace{3.5 cm} Signatur nyckelansvarig:

\newpage

\subsection*{Härmed bekräftar jag att jag har fått tillbaka depositionen samt att nyckeln är återlämnad}
\vspace{1cm}
Datum:
\vspace{2cm} \\
Namn nyckellånare
\hspace{4.7cm}
Namn nyckelansvarig
\vspace{2cm} \\
Signatur nyckellånare
\hspace{4.3cm}
Signatur nyckelansvarig
\vspace{1cm}\\
\_\_\_\_\_\_\_\_\_\_\_\_\_\_\_\_\_\_\_\_\_\_ \hspace{1.5 cm} \_\_\_\_\_\_\_\_\_\_\_\_\_\_\_\_\_\_\_\_\_\_
\vspace{5cm} \\
Övrigt:

\newpage
\title{Nyckelavtal - Bilaga 1}
\authors{Uppdaterad 2023}

%\makeheadfoot
\makesimpletitle
\vspace{-2 cm}
Allmänt ansvar när du har nycklar är att du städar efter dig när du vistats i lokalerna och att du lägger undan dina saker på ett bra sätt i de förvaringsutrymmen du och din kommitté/förening har. Om detta inte sköts kommer en varning ges och därefter kan du förlora din nyckelföremån eller att du och din kommitté/förening förlorar förvaringsutrymmen.

Väljer du att låna ut din nyckel ska det endast ske till andra aktiva IT-teknologer som inte blivit av med sina nyckelprivilegier på grund av misskötsel. Lånar du ut nyckeln är du ytterst ansvarig för att lokalerna nyckeln har tillgång till lämnas i samma eller bättre skick.
\end{document}